%%%%%%%%%%%%%%%%%%%%%%%%%%%%%%%%%%%%%%%%%%%%%%%%%%%%%%%%%%%
%% Diese Datei sollten Sie nicht anpassen, sie definiert %%
%% ein Dokument der Form, in der wir es erwarten.        %%
%%%%%%%%%%%%%%%%%%%%%%%%%%%%%%%%%%%%%%%%%%%%%%%%%%%%%%%%%%%

%Schriftgröße, ein- oder zweiseitig, Papierformat, Dokumententyp
\documentclass[12pt,oneside,a4paper]{scrartcl}

\usepackage[utf8]{inputenc}
\usepackage[T1]{fontenc}

%Seitenränder
\usepackage[left=2.5cm,right=2.5cm,top=2.5cm,bottom=2cm]{geometry}

%NDR und Umlaute
\usepackage{ngerman}

%Kopf- und Fußzeile
\usepackage{fancyhdr}
\pagestyle{fancy}
\fancyhf{}

%Kopfzeile links bzw. innen
\fancyhead[L]{\name}

%Kopfzeile mittig
\fancyhead[C]{\thepage}

%Kopfzeile rechts bzw. außen
\fancyhead[R]{\thema}

%Linie oben
\renewcommand{\headrulewidth}{0.5pt}

%Für farbige Links
\usepackage{color}

%Hübsche Schriften im PDF-Viewer
\usepackage{ae}
\usepackage{times}

% Brauchbare PDF-Links und angaben im PDF-Header
\usepackage[pdftex,
 raiselinks=true,%
  bookmarks=true,%
  colorlinks=false,% Gibt man keine gedruckte Version ab, sondern das PDF, sollte man erwägen diesesn Wert auf "true" zu ändern
  bookmarksopenlevel=1,%
  bookmarksopen=true,%
  bookmarksnumbered=true,%
  hyperindex=true,% 
  plainpages=false,% correct hyperlinks
  pdfpagelabels=true,% view TeX pagenumber in PDF reader
 pdfstartview=FitH]{hyperref}
%%  pdfborder={0 0 0.5}
 %%  pdfauthor={\name},
%%  pdfsubject={\veranstaltung},
%%  pdfkeywords={\keywords},
%%  pdftitle={\thema},
 
%Thumbnails im PDF
\usepackage{thumbpdf}

%hübschere Tabellenabstände
\usepackage{booktabs}

%diverser mathematischer Kram
\usepackage{amsmath}

% Für den dinat zitier stil
\usepackage{natbib}

% Graphiken
\usepackage[final]{graphicx}

% Verhindern von "Schusterjungen" und "Hurenkindern"
\clubpenalty = 10000
\widowpenalty = 10000
\displaywidowpenalty = 10000
\tolerance=500 %Zeilenumbruch


