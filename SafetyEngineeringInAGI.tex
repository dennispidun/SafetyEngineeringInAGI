%%%%%%%%%%%%%%%%%%%%%%%%%%%%%%%%%%%%%%%%%%%%%%%%%%%%%%%%%%%%%%%%%%%%%%%%%
% Angaben zur Person
%%%%%%%%%%%%%%%%%%%%%%%%%%%%%%%%%%%%%%%%%%%%%%%%%%%%%%%%%%%%%%%%%%%%%%%%%
\newcommand{\name}{Dennis Pidun}
\newcommand{\matr}{??????}
\newcommand{\email}{pidund@uni-hildesheim.de}
\newcommand{\studgang}{Angewandte Informatik (B.Sc.)}
\newcommand{\thema}{Safety Engineering in AGI}
\newcommand{\keywords}{}
\newcommand{\betreuer}{Dr. Pascal Reu�}

%%%%%%%%%%%%%%%%%%%%%%%%%%%%%%%%%%%%%%%%%%%%%%%%%%%%%%%%%%%%%%%%%%%%%%%%%
% Angaben zum Seminar
%%%%%%%%%%%%%%%%%%%%%%%%%%%%%%%%%%%%%%%%%%%%%%%%%%%%%%%%%%%%%%%%%%%%%%%%%
\newcommand{\semester}{Wintersemester 2019/2020}
\newcommand{\vtyp}{Seminar}
\newcommand{\veranstaltung}{IIS Seminar}
\newcommand{\prof}{Prof. Dr. Klaus-Dieter Althoff}
\newcommand{\lehrstuhl}{Institut f�r Informatik\\Bereich Intelligente Informationssysteme}

%%%%%%%%%%%%%%%%%%%%%%%%%%%%%%%%%%%%%%%%%%%%%%%%%%%%%%%%%%%
%% Diese Datei sollten Sie nicht anpassen, sie definiert %%
%% ein Dokument der Form, in der wir es erwarten.        %%
%%%%%%%%%%%%%%%%%%%%%%%%%%%%%%%%%%%%%%%%%%%%%%%%%%%%%%%%%%%

%Schriftgr��e, ein- oder zweiseitig, Papierformat, Dokumententyp
\documentclass[12pt,oneside,a4paper]{scrartcl}

\usepackage[latin1]{inputenc}
%Seitenr�nder
\usepackage[left=2.5cm,right=2.5cm,top=2.5cm,bottom=2cm]{geometry}

%NDR und Umlaute
\usepackage{ngerman}
\usepackage[latin1]{inputenc}

%Kopf- und Fu�zeile
\usepackage{fancyhdr}
\pagestyle{fancy}
\fancyhf{}

%Kopfzeile links bzw. innen
\fancyhead[L]{\name}

%Kopfzeile mittig
\fancyhead[C]{\thepage}

%Kopfzeile rechts bzw. au�en
\fancyhead[R]{\thema}

%Linie oben
\renewcommand{\headrulewidth}{0.5pt}

%F�r farbige Links
\usepackage{color}

%H�bsche Schriften im PDF-Viewer
\usepackage{ae}
\usepackage{times}

% Brauchbare PDF-Links und angaben im PDF-Header
\usepackage[pdftex,
 raiselinks=true,%
  bookmarks=true,%
  colorlinks=false,% Gibt man keine gedruckte Version ab, sondern das PDF, sollte man erw�gen diesesn Wert auf "true" zu �ndern
  bookmarksopenlevel=1,%
  bookmarksopen=true,%
  bookmarksnumbered=true,%
  hyperindex=true,% 
  plainpages=false,% correct hyperlinks
  pdfpagelabels=true,% view TeX pagenumber in PDF reader
 pdfstartview=FitH]{hyperref}
%%  pdfborder={0 0 0.5}
 %%  pdfauthor={\name},
%%  pdfsubject={\veranstaltung},
%%  pdfkeywords={\keywords},
%%  pdftitle={\thema},
 
%Thumbnails im PDF
\usepackage{thumbpdf}

%h�bschere Tabellenabst�nde
\usepackage{booktabs}

%diverser mathematischer Kram
\usepackage{amsmath}

% F�r den dinat zitier stil
\usepackage{natbib}

% Graphiken
\usepackage[final]{graphicx}

% Verhindern von "Schusterjungen" und "Hurenkindern"
\clubpenalty = 10000
\widowpenalty = 10000
\displaywidowpenalty = 10000
\tolerance=500 %Zeilenumbruch




%%%%%%%%%%%%%%%%%%%%%%%%%%%%%%%%%%%%%%%%%%%%%%%%%%%%%%%%%%%%%%%%%%%%%%%%%
% Zusatzpakete
%%%%%%%%%%%%%%%%%%%%%%%%%%%%%%%%%%%%%%%%%%%%%%%%%%%%%%%%%%%%%%%%%%%%%%%%%

\begin{document}
%%%%%%%%%%%%%%%%%%%%%%%%%%%%%%%%%%%%%%%%%%%%%%%%%%%%%%%%%%
%% Dies hier ist die Titelseite.                        %%
%% Auch hier brauchen sie KEINE �nderungen vorzunehmen! %%
%% Die entsprechenden Angaben werden aus den Variablen  %%
%% in IIS-Seminar-Vorlage.tex �bernommen.               %%
%%%%%%%%%%%%%%%%%%%%%%%%%%%%%%%%%%%%%%%%%%%%%%%%%%%%%%%%%%

\thispagestyle{empty}
\linespread {1.25}\selectfont % eineinhalbfachen Zeilenabstand f�r diesen Block
\begin{flushright}
Universit\"at Hildesheim\\
\lehrstuhl\\
\prof\\
\end{flushright}
\begin{center}
\linespread {1.05}\selectfont % 1.25-facher Zeilenabstand
\vfill
\LARGE{\vtyp\\*[.1cm]\parbox{0.6\textwidth}{\begin{center}\veranstaltung\end{center}}\\*[.2cm]}
\large{\textbf{Thema: \thema\\~\\}
Betreuer: \betreuer\\~\\
\semester}
\vfill
\name\\
Matrikelnummer: \matr\\
Studiengang: \studgang\\~\\
E-Mail: \href{mailto:\email}{\email}\\
\end{center}
% R�cksetzen des Seitenz�hlers
\setcounter{page}{0}
\newpage


%%%%%%%%%%%%%%%%%%%%%%%%%%%%%%%%%%%%%%%%%%%%%%%%%%%%%%%%%%%%%%%%%%%%%%%%%
% Abstract etc.
%%%%%%%%%%%%%%%%%%%%%%%%%%%%%%%%%%%%%%%%%%%%%%%%%%%%%%%%%%%%%%%%%%%%%%%%%
\section*{Abstract}
\begin{abstract}\textsl{
Abstract einf�gen.
}\end{abstract}
\newpage

%%%%%%%%%%%%%%%%%%%%%%%%%%%%%%%%%%%%%%%%%%%%%%%%%%%%%%%%%%%%%%%%%%%%%%%%%
% Inhaltsverzeichnis
%%%%%%%%%%%%%%%%%%%%%%%%%%%%%%%%%%%%%%%%%%%%%%%%%%%%%%%%%%%%%%%%%%%%%%%%%
\tableofcontents

%%%%%%%%%%%%%%%%%%%%%%%%%%%%%%%%%%%%%%%%%%%%%%%%%%%%%%%%%%%%%%%%%%%%%%%%%
% Inhalt
%%%%%%%%%%%%%%%%%%%%%%%%%%%%%%%%%%%%%%%%%%%%%%%%%%%%%%%%%%%%%%%%%%%%%%%%%

\section{Einf�hrung}
    \subsection{Motivation}
    \subsection{Machine Ethics}
    Um die Frage, warum Machine Ethics wichtig sind zu kl�ren, stellen wir uns zun�chst ein
    Scenario vor, welches bereits Patrick Lin \cite[p. 70]{maurer_gerdes_lenz_winner_2015}
    aufgestellt wurde. Hierbei handelt es sich um eine Situation, welche bereits heute auf den 
    Stra�en auftreten kann. Um die moralischen Entscheidungen in dem Prozess des Fahrens zu verdeutlichen, 
    stellen wir uns also vor, dass man in einem Auto sitzt und nun eine Entscheidung treffen muss. N�mlich
    die Entscheidung, ob man nach links ausweicht und damit ein junges M�dchen umbringt oder ob 
    man nach rechts ausweicht und damit eine �ltere Seniorin umbringt. Weicht man nicht aus, werden
    augenblicklich beide Menschen mit in den Tot gerissen. Dieses Beispiel ist tats�chlich sehr 
    dramatisch gew�hlt, was jedoch sehr gut verdeutlicht, welche nicht rationalen Entscheidungen 
    getroffen werden m�ssen. Egal welche Entscheidung hierbei getroffen wird, am Ende stirbt mindestens
    ein Mensch, was in jeder Hinsicht einen Verlust darstellt und ohnehin moralisch nicht korrekt w�re. 
    \cite[p. 70]{maurer_gerdes_lenz_winner_2015}
    Mit seiner Entscheidung kann man letztendlich nur bestimmen, welche Person benachteiligt wird. In einer 
    solchen Situation jedoch h�tte der Mensch ohnehin keine Zeit diese Entscheidung rational zu betrachten. 
    Ferner h�tte der Mensch ebenfalls nicht die M�glichkeiten und das Wissen gerecht zu entscheiden.
    
    Ein weiteres Problem w�re au�erdem, dass man nicht genau wei�, welche moralischen Werte man verankern soll. 
    \cite[p. 1]{yampolskiy2013safety} Es wird daher viel diskutiert, welche moralischen Wertvorstellungen die 
    richtigen sind. In diversen Literaturen findet man unter verschiedenen Titeln immer wieder Diskussionen,
    welche sich genau mit dieser Fragestellung auseinandersetzen. Yampolskiy spricht hier davon, dass keine
    effektiven Ma�nahmen getroffen werden, da sich einerseits viel damit besch�ftigt wird, sich andererseits
    aber nichts wandelt und kein nutzbares Ergebnis herauskommt. \cite[p. 1]{yampolskiy2013safety}
    
    
    Warum sind moralische Werte wichtig?
    Wie handelt eine Maschine ohne moralischen Vorstellungen?
    Wann handelt eine Maschine richtig?
    Wie genau pr�gt sich ein falsches Bild ein?
    Warum ist eine moralische Instanz bei der Auswertung gro�er Datenmengen n�tig? 
    K�nnen moralische Richtlinien helfen bzw. unterst�tzen?
    
    
\section{Artificial General Intelligence}
    \subsection{Classical Artificial Intelligence}
    \subsection{Verbindung zu ''Strong AI''}
    \subsection{Aktueller Forschungsstand bei AGI-Systems}
\section{Safety Engineering}
    \subsection{Probleme in AI Engineering}
    \subsection{AI Safety in klassischen AI Systems}
    \subsection{AI Safety in Artificial Superintelligent Systems}

%% Eher nochmal bearbeiten:
    \subsection{Probleml�sungsans�tze} 
    \subsection{Simulated Areas / Simulations}
\section{The Artificial Confinementproblem}
    \subsection{Kritik des Confinement Approach}
    \subsection{M�gliche Escape Paths}
    \subsection{Social Engineering Attacks}
\section{AI Boxing Strategies}
    \subsection{Physical Boxing}
    \subsection{Psychological Boxing}
    \subsection{Kombination mit anderen Limitierungstechniken}
\section{Fazit}












\cite{lol}

\newpage
%%%%%%%%%%%%%%%%%%%%%%%%%%%%%%%%%%%%%%%%%%%%%%%%%%%%%%%%%%%%%%%%%%%%%%%%%
%% Einbinden der Quellen
%% https://www.overleaf.com/learn/latex/Bibliography_management_with_bibtex#Reference_guide
%%%%%%%%%%%%%%%%%%%%%%%%%%%%%%%%%%%%%%%%%%%%%%%%%%%%%%%%%%%%%%%%%%%%%%%%%
\addcontentsline{toc}{section}{\bibname}
\bibliography{quellen}
\bibliographystyle{dinat}

\end{document}
